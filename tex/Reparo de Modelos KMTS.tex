%% Modelo ABNT 
\documentclass[normaltoc,capchap,capsec,times]{abnt}
\usepackage[hang,flushmargin]{footmisc} 
\usepackage[brazil]{babel} % Idioma do documento
\usepackage[utf8]{inputenc}
\usepackage{abnt-alf}
\usepackage[alf]{abntcite}
\usepackage[T1]{fontenc}
\usepackage{graphicx}
\usepackage[font=footnotesize]{caption,subcaption}
\usepackage{chngcntr}
\counterwithout{figure}{chapter}
\usepackage{amsfonts}
\usepackage{subcaption}
\usepackage{amsmath}
\usepackage{longtable}
\usepackage{listings}
\usepackage{enumitem}

\usepackage{tikz}
\usetikzlibrary{backgrounds,fit,calc,shapes}
\usetikzlibrary{arrows}

\usepackage{pgfplots}
\usepackage[acronym]{glossaries}
\usepackage[paginas=nao]{tabela-simbolos}
\usepackage{amsthm}
\usepackage{url}

\lstset{extendedchars=\true}

\newtheorem{definition}{Definição}[section]
\newtheorem{lemma}{Lema}[section]
\newtheorem{theorem}{Teorema}[section]


\makeindex

%% Fim do Preâmbulo
%% --------------------------------------------------------------------------------------
%% Identificação:

% Título do Trabalho
\title{Reparo de Modelos KMTS}
\newcommand{\meutitulo}{Reparo de Modelos KMTS}

\local{Salvador}

% Data da defesa
\date{23 de Dezembro de 2014}

% Autor
\author{Pedro Dumont Marques}

\comentario{Monografia apresentada ao Curso de
graduação em Ciência da Computação,
Departamento de Ciência da Computação,
Instituto de Matemática, Universidade Federal
da Bahia, como requisito parcial para
obtenção do grau de Bacharel em Ciência da
Computação. \\ Orientadora: Profª Drª. Aline Maria Santos Andrade.}

%% Capa:
\let\oldteste\teste
\let\oldcapa\capa
\renewcommand{\capa}{

\begin{center}
\includegraphics[width=0.2\textwidth,natwidth=610,natheight=642]{imagens/brasao_ufba.jpg}
\\
\centering{ 
	\bf{
	    \LARGE{
			\uppercase{UNIVERSIDADE FEDERAL DA BAHIA} \\
 	    }
	    \Large {
          	\uppercase{INSTITUTO DE MATEMÁTICA} \\
	    }
    	\large {
           	\uppercase{DEPARTAMENTO DE CIÊNCIA DA COMPUTAÇÃO} \\
        }
    }           
}
\end{center}

\vfill

\begin{center}
    \bf{
       \large{Pedro Dumont Marques  \\  }
    }
\end{center}

\vspace{0.2in}

\begin{center}
	\bf{
       \LARGE{ \textit{ \meutitulo } } \\
   	}
\end{center}

\vfill
	\hspace{\stretch{1}}
   	\vfill
   	\begin{center}
      \normalsize{
          Salvador \\
          2014
       }
   	\end{center}
}

%% ----------------------------------------------------------------------------------
%% Início do documento
%% ----------------------------------------------------------------------------------

\begin{document}
\capa 
\folhaderosto

% Resumo em Português
% Se preferir, crie um arquivo à parte e o inclua via \include{}
\begin{resumo}
<DIGITE O RESUMO AQUI>

% Palavras-chave do resumo em Português
\textbf{Palavras-chave}: Métodos Formais, Verificação de Modelos, Estruturas de Kripke, KMTS, Revisão de Modelos, Reparo de Modelos.
\end{resumo}

% Resumo em Inglês
% Se preferir, crie um arquivo à parte e o inclua via \include{}
\begin{abstract}

% Palavras-chave do resumo em Inglês
\textbf{Key-words}: Formal Methods, Model Checking, Kripke Structures, KMTS, Model Revision, Model Repair.
\end{abstract}


% Sumário
% Comente para ocultar
% \tableofcontents
\sumario

% Lista de figuras
% Comente para ocultar
\listoffigures

% Lista de tabelas
% Comente para ocultar
% \listoftables

\renewcommand{\listofabreviationsname}{Lista de Abreviaturas e Siglas}
\listadesiglas

%% --------------------------------------------------------------------------------------
%% Parte Textual
%% Capítulos
%% --------------------------------------------------------------------------------------

% É aconselhável criar cada capítulo em um arquivo à parte, digamos
% "capitulo1.tex", "capitulo2.tex", ... "capituloN.tex" e depois
% incluí-los com:
% \include{capitulo1}
% \include{capitulo2}
% ...
% \include{capituloN}

%% Introdução ---------------------------------------------------------------------------
\chapter{Introdução}

% TODO: Descrever detalhadamente a motivação (Reparo de Modelos), o contexto (metodos formais, agentes, revisão de crenças), o objetivo (a implementação) e o conteúdo

A área de métodos formais tem sido cada vez mais visada para a produção de softwares. A criação de modelos para o sistema, nos momentos iniciais de levantamento de requisitos, e sua póstuma automatização em sistemas robustos e formalmente verificados, são de grande interesse, uma vez que são aceleradores e facilitadores de desenvolvimento. Porém, um grande problema surge quando se trata de levantamento de requisitos, a incompletude do modelo do sistema.

Nos estágios iniciais, não se sabe muito das propriedades que o sistema deve atender, ou mesmo se este atenderá corretamente as propriedades especificadas. Obviamente não temos interesse em indefinições ou incorreções no nossos modelos, e estas devem ser tratadas.
Buscando a automação do processo de correção de modelos, nós podemos olhar para o campo da inteligência artificial, com agentes inteligentes que possuem uma base de crenças.

Uma base de conhecimento ou crenças nada mais é que um conjunto de propriedades no qual aquele agente acredita e reconhece como verdadeiras para seu mundo. Nós podemos ver esse conjunto de propriedades como um modelo para um sistema, como o próprio sistema do agente. Ao explorar o mundo ao seu redor, um agente acaba por analisar e descobrir novas informações e propriedades a serem acrescentadas às suas crenças. Porém o que fazer quando essas informações são conflitantes? Nós não queremos que nosso pequeno e confuso agente destrua todas suas crenças para acomodar a nova informação. Por isso sempre buscamos as soluções que menos modifiquem sua base de crenças, soluções minimais.

Com esse objetivo em mente, toda uma área de verificação de modelos foi desenvolvida, utilizando-se de linguagens formais e lógicas para especificação e abstração de propriedades, e algoritmos de verificação e revisão. Os modelos do sistema são representados aqui em estruturas de Kripke, grafos direcionados contendo informações sobre estados do sistema, propriedades e transições.

Em \cite{aline} foi proposto o uso de Estruturas Modais de Kripke (\textit{Kripke Modal Transition System} -- KMTS)\sigla{KMTS}{Kripke Modal Transition System} para representar informações parciais sobre o sistema, como opções do fluxo de operações e indefinições de propriedades ou estados. Um KMTS pode ser visto como uma abstração de um conjunto de estruturas de Kripke, cada estrutura representando uma das possibilidades geradas pelas informações parciais do KMTS.

Para realizar a verificação dos modelos sobre as estruturas de Kripke, a linguagem do $\mu$-calculus é a preferida e largamente utilizada pela sua completude. Outras lógicas podem ser facilmente derivadas do $\mu$-calculus. Para o nosso escopo, porém, de sistemas computacionais, o $\mu$-calculus pode ser completo até demais. Por isso foi proposto e utilizado a Lógica de Árvore Computacional (CTL - \textit{Computing Tree Logic})\sigla{CTL}{Computing Tree Logic} para as propriedades dos modelos a serem verificadas.

Como representamos indefinições e informações parciais no nossos modelos KMTS, a lógica utilizada deve possuir esse terceiro valor, indefinido $\perp$, resultando em algoritmos de verificação de modelos de três valores. Na nossa abordagem, é utilizado um algoritmo baseado em jogos de três valores, proposto em \cite{grumberglosing}.

Então, quando nosso pequeno agente realiza a verificação do seu modelo de acordo com a nova informação obtida, ele pode obter três resultados diferentes. $\top$ se essa informação está okay com seu modelo de crenças; $\perp$ se essa informação está indefinida, ou ainda resultando em um estado ou fluxo indefinido; e $F$ se essa informação contradiz algo em seu modelo.

Ele deverá fazer agora uma revisão de seu modelo de crenças, procurando assimilar, ou satisfazer, essa nova propriedade com o mínimo de mudanças possíveis. Um algoritmo para essa revisão automática foi inicialmente proposto em \cite{wasserman} e posteriormente evoluído em \cite{aline}, envolvendo a abordagem de jogos de três valores de \cite{grumberglosing}. Esse algoritmo é dividido em duas partes, refinamento e reparo do modelo, onde o objetivo é revisar o modelo quando a verificação deste retorna $\perp$ e $F$, respectivamente.

\cite{jandson} realizou a implementação do refinamento automático do modelo, expandindo o algoritmo com o uso de propriedades de grafos, resultando em um programa completamente funcional. Ainda neste trabalho foi realizada a implementação de algoritmos auxiliares, como o de verificação de modelos com lógica de três valores baseado em jogos.

O objetivo deste trabalho é a realização da implementação do reparo do modelo KMTS para quando a verificação de modelos retorna $F$, expandindo-se no algoritmo proposto em \cite{aline} e no de refinamento implementado. Fazemos também considerações sobre questões de minimalidade das soluções e propostas para melhorias da complexidade do algoritmo.

A estrutura dessa monografia segue a seguinte forma: No capítulo \ref{cap:verificacao} apresentamos a base teórica com as definições da verificação de modelos. O algoritmo de verificação de modelos usando CTL de três valores sobre KMTS é apresentado no capítulo \ref{cap:revisao}, com exemplos do refinamento. Já no capítulo \ref{cap:reparo}, falamos sobre o algoritmo de reparo, mostrando seu funcionamento. A implementação e estratégias utilizadas são apresentadas no capítulo \ref{cap:implementacao}, enquanto que uma evolução das mesmas e o conceito de minimalidade é explorado no capítulo \ref{cap:minimalidade}. Finalmente, no capítulo \ref{cap:conclusao} traçamos as considerações finais e apresentamos os trabalhos futuros.

%%
%% Capítulo Verificação -----------------------------------------------------------------
%%
\chapter{Verificação de Modelos}
\label{cap:verificacao}

TODO: Entrar no âmbito de modelos, revisando o background necessário para o entendimento do trabalho. Falar sobre estruturas de Kripke, KMTS, CTL vs mu-calculos, model checking com jogos.

\textbf{devo redefinir tudo e explicar o que é estrutura de kripke, mostrar exemplos, kmts, ctl e etc?}

Verificação de Modelos é uma área robusta e em expansão de Métodos Formais, cujo objetivo é a validação e garantia da corretude e outras propriedades de modelos. Existe uma gama de algorítmos de verificação de modelos, usados com diversas estruturas de dados e lógicas. Neste trabalho focaremos em dois modelos formais, estruturas de Kripke e KMTS.

\begin{definition}
\textnormal{
Uma estrutura de Kripke é uma tupla $ M = (S, \to , L)$ onde $S$ é um conjunto de estados finitos, $\to$ é uma relação binária em S, $\to \in S^2$, tal que para todo $s \in S$, existe algum $s' \in S$ com $s \to s'$. $L$ é uma função de rotulação $L : S \to AP$, onde $AP$ é um conjunto de fórmulas atômicas. A função rotula os átomos que ocorrem em um determinado estado $s$. 
}
\end{definition}

Uma estrutura de Kripke pode ser representada como um grafo direcionado, muito similar a autômatos, contendo informações dos estados do sistema e suas propriedades, como na figura \ref{fig:ex_kripke1}. Mapeando também ações que levam a mudança de estados, como transições do grafo.

\begin{figure}[htb]
\begin{center}
 \begin{tikzpicture}[->,shorten >=2pt,auto,node distance=2.5cm,align=center,
  thick,every node/.style={circle,fill=blue!18,draw,minimum size=2.5em, font=\sffamily\small\bfseries}]

  \node[ label=120:$s_0$ ] (0) { $ p , q$ };
  \node[ label=(75):$s_2$ ] (2) [below right of=0] {$ r$};
  \node[ label=(120):$s_1$ ] (1) [below  left of=0] {$ q , r$ };

  \path[every node/.style={font=\sffamily\small}]
    (0) edge node [left] {} (2)
    (0) edge [bend right] node[right] {} (1)        
    (1) edge node [right] {} (2)
    (1) edge node [right] {} (2)
        edge [bend right] node[right] {} (0)
    (2) edge [loop right] node {} (2)       
    ;

\end{tikzpicture}
\end{center}
\caption[Estrutura de Kripke representado por um grafo direcionado]{Estrutura de Kripke representado por um grafo direcionado. \\ Fonte. \citeonline[p. 136]{huth}}
\label{fig:ex_kripke1}
\end{figure}

Um KMTS é similar a uma estrutura simples de Kripke, porém tem o poder expressivo para representar informações parciais, na forma das transições \textit{may} e estados com literais indefinidos. Enquanto que em uma estrutura de Kripke a omissão da rotulação de um literal significa que este literal é negativo, no KMTS isso significa que o literal é indefinido, sendo necessária a rotulação também de quando os literais forem negativos naquele estado.

\begin{definition}\cite{aline}
\textnormal{KMTS é uma tupla $M = < AP$, $S,$ $S_o,$  $R^+,$ $R^-,$  $ L$ $ > $, onde $S$ é um conjunto de estados finitos, $S_0 \subseteq S$ é o conjunto de estados iniciais, $R^+ \subseteq S \times S$ e $R^- \subseteq S \times S$  são relações de transição, tal que $R^+ \subseteq R^-$; $Lit = {AP \cup \{ \neg p | p \in AP\}}$ é o conjunto de literais sobre AP e $L: S \to 2^{Lit}$ é uma função de rotulação, tal que para todos os estados $s$ e $p \in AP$, no máximo somente um entre $p$ e $\neg p$ ocorre. As transições $R^+$ e $R^-$ correspondem às transições \textit{must} e \textit{may}, respectivamente.
}
\end{definition}

É fácil e interessante perceber que cada indefinição em um KMTS representa uma estrutura de Kripke diferente. Por exemplo, cada transição \textit{may} representa uma estrutura de Kripke onde essa transição é \textit{must} e uma estrutura onde essa transição não existe. O mesmo vale para indefinições em estados. Cada omissão de um determinado literal em um estado representa duas estruturas de Kripke, uma onde o literal é positivo e outra onde é negativo.

Esse poder de abstração do KMTS foi atraente para o nosso objetivo, podendo realizar a verificação e revisão em um conjunto de modelos de uma forma muito mais eficiente

%%
%% Capítulo Revisão -------------------------------------------------------------------
%%
\chapter{Revisão de Modelos}
\label{cap:revisao}

TODO: Falar sobre a revisão de modelos KMTS, sobre o que já foi feito na área, abordagens utilizadas, sobre o refinamento já implementado.


%%
%% Capítulo Reparo --------------------------------------------------------------------
%%
\chapter{Reparo de Modelos}
\label{cap:reparo}

TODO: Falar sobre o reparo, toda a parte teórica. Sobre a construção das jogadas, o que torna o resultado do model checking falso, a necessidade da correção, os conceitos de minimalidade já vistos e os cuidados com o reparo.Explicar o algoritmo por alto.

%%
%% Capítulo Implementação ---------------------------------------------------------------
%%
\chapter{Implementação do Reparo de Modelos}
\label{cap:implementacao}

TODO: Falar sobre a implementação do algoritmo em si. Estratégias utilizadas, linguagem, aplicação de tecnicas para facilitar o desenvolvimento, melhorias sobre o algoritmo, tratos de eficiencia e talvez sobre a GUI se for feita.

%%
%% Capítulo Minimalidade ----------------------------------------------------------------
%%
\chapter{Inclusão de Estados e Minimalidade}
\label{cap:minimalidade}

TODO: Falar sobre a operaçao p6, a inclusão de estados, para o caso de jogadas EX Y em que não existam mais transições. O que fazer, os impactos de criações de estados, de novas transições. p6 afeta mais a minimalidade do que p2 e p4. Vantagens e desvantagens de sua aplicação. 

%%
%% Conclusão ----------------------------------------------------------------------------
%%
\chapter{Conclusão}
\label{cap:conclusao}

TODO: Concluir o trabalho, citar as experiencias aprendidas, os artefatos criados e a contribuição realizada para a área. Trabalhos futuros que possam surgir de ou a partir deste trabalho.

%%
%% Parte pós-textual
%%

% Apêndices
% Comente se não houver apêndices
%\appendix

% É aconselhável criar cada apêndice em um arquivo à parte, digamos
% "apendice1.tex", "apendice.tex", ... "apendiceM.tex" e depois
% incluí-los com:
% \include{apendice1}
% \include{apendice2}
% ...
% \include{apendiceM}


% Bibliografia
% É aconselhável utilizar o BibTeX a partir de um arquivo, digamos "biblio.bib".
\bibliographystyle{abnt-alf}
\bibliography{bibliografia}

%% Fim do documento
\end{document}